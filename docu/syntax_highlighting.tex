\definecolor{mygreen}{rgb}{0,0.6,0}
\definecolor{mygray}{rgb}{0.5,0.5,0.5}
\definecolor{mylightgray}{rgb}{0.93,0.93,0.93}
\definecolor{mymauve}{rgb}{0.58,0,0.82}
\definecolor{darkblue}{rgb}{0.0,0.0,0.6}
\definecolor{cyan}{rgb}{0.0,0.6,0.6}
\definecolor{lightgray}{rgb}{.9,.9,.9}
\definecolor{darkgray}{rgb}{.4,.4,.4}
\definecolor{purple}{rgb}{0.65, 0.12, 0.82}
\definecolor{black}{rgb}{0,0,0}
\definecolor{white}{rgb}{1,1,1}
\definecolor{magicmint}{rgb}{0.67, 0.94, 0.82}

\lstset{ 
	backgroundcolor=\color{white},   % choose the background color; you must add \usepackage{color} or \usepackage{xcolor}; should come as last argument
	basicstyle=\footnotesize,        % the size of the fonts that are used for the code
	breakatwhitespace=false,         % sets if automatic breaks should only happen at whitespace
	breaklines=true,                 % sets automatic line breaking
	captionpos=t,                    % sets the caption-position to bottom
	commentstyle=\color{mygreen},    % comment style
	deletekeywords={...},            % if you want to delete keywords from the given language
	escapeinside={\%*}{*)},          % if you want to add LaTeX within your code
	extendedchars=true,              % lets you use non-ASCII characters; for 8-bits encodings only, does not work with UTF-8
	firstnumber=1,                % start line enumeration with line 1000
	frame=single,	                   % adds a frame around the code
	keepspaces=true,                 % keeps spaces in text, useful for keeping indentation of code (possibly needs columns=flexible)
	keywordstyle=\color{blue},       % keyword style
	morekeywords={*,...},            % if you want to add more keywords to the set
	numbers=left,                    % where to put the line-numbers; possible values are (none, left, right)
	numbersep=2pt,                   % how far the line-numbers are from the code
	numberstyle=\tiny\color{mygray}, % the style that is used for the line-numbers
	rulecolor=\color{black},         % if not set, the frame-color may be changed on line-breaks within not-black text (e.g. comments (green here))
	showspaces=false,                % show spaces everywhere adding particular underscores; it overrides 'showstringspaces'
	showstringspaces=false,          % underline spaces within strings only
	showtabs=false,                  % show tabs within strings adding particular underscores
	stepnumber=1,                    % the step between two line-numbers. If it's 1, each line will be numbered
	stringstyle=\color{mymauve},     % string literal style
	tabsize=2,	                   % sets default tabsize to 2 spaces
	title=\lstname,                   % show the filename of files included with \lstinputlisting; also try caption instead of title
	xleftmargin=2em,
	framexleftmargin=2em,
	literate=%
	{Ö}{{\"O}}1
	{Ä}{{\"A}}1
	{Ü}{{\"U}}1
	{ß}{{\ss}}1
	{ü}{{\"u}}1
	{ä}{{\"a}}1
	{ö}{{\"o}}1
	{~}{{\textasciitilde}}1
}
%----------------------------------------------------------------------------------------


\lstdefinestyle{customJava}{ %
  backgroundcolor=\color{white},   % choose the background color
  basicstyle=\footnotesize,        % size of fonts used for the code
  breaklines=true,                 % automatic line breaking only at whitespace
  commentstyle=\color{mygreen},    % comment style
  escapeinside={\%*}{*)},          % if you want to add LaTeX within your code
  keywordstyle=\color{blue},       % keyword style
  stringstyle=\color{mymauve},     % string literal style
  showspaces=false,
  showstringspaces=false,
  tabsize=2,
  numbers=left,
  stepnumber=1,
  frame=single,
  xleftmargin=2em,
  framexleftmargin=2em
}

\lstdefinelanguage{JavaScript}{
  keywords={break, case, catch, continue, debugger, default, delete, do, else, false, finally, for, function, if, in, instanceof, new, null, return, switch, this, throw, true, try, typeof, var, void, while, with},
  morecomment=[l]{//},
  morecomment=[s]{/*}{*/},
  morestring=[b]',
  morestring=[b]",
  ndkeywords={class, export, boolean, throw, implements, import, this},
  keywordstyle=\color{blue}\bfseries,
  ndkeywordstyle=\color{darkgray}\bfseries,
  identifierstyle=\color{black},
  commentstyle=\color{purple}\ttfamily,
  stringstyle=\color{red}\ttfamily,
  sensitive=true
}

\lstset{
   language=JavaScript,
   backgroundcolor=\color{lightgray},
   extendedchars=true,
   basicstyle=\footnotesize\ttfamily,
   showstringspaces=false,
   showspaces=false,
   numbers=left,
   numberstyle=\footnotesize,
   numbersep=9pt,
   tabsize=2,
   breaklines=true,
   showtabs=false
}

\lstdefinelanguage{Text}
{ 
	sensitive=false 
}

\lstdefinelanguage{XML}
{
  morestring=[b]",
  morestring=[s]{>}{<},
  morecomment=[s]{<?}{?>},
  stringstyle=\color{black},
  identifierstyle=\color{darkblue},
  keywordstyle=\color{cyan},
  morekeywords={xmlns,version,type}% list your attributes here
}